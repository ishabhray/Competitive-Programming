%%%%%%%%%%%%%%%%%
% This is altacv.cls (v1.1.5, 1 December 2018) written by
% LianTze Lim (liantze@gmail.com).
%
%% It may be distributed and/or modified under the
%% conditions of the LaTeX Project Public License, either version 1.3
%% of this license or (at your option) any later version.
%% The latest version of this license is in
%%    http://www.latex-project.org/lppl.txt
%% and version 1.3 or later is part of all distributions of LaTeX
%% version 2003/12/01 or later.
%%
%%
% Contributions:
% - https://github.com/akreuzer Added ragged2e option (5 Nov 2018)
% - https://github.com/stefanogermano Fixed bad boxes and undefined font shape (July 2018)
% - https://github.com/foohyfooh Fixed blank spaces in \cvevent and bad link in README.md (June 2018)

%%%%%%%%%%%%%%%%
\NeedsTeXFormat{LaTeX2e}[1995/12/01]
%% v1.4: Detect TL2018 to handle accented characters in class information
\@ifl@t@r\fmtversion{2018/04/01}{\UseRawInputEncoding}{}
\ProvidesClass{altacv}[2018/12/01 AltaCV v1.1.5, yet another alternative class for a résumé/curriculum vitae.]

%% v1.1: Optionally load academicons
\newif\if@academicons
\DeclareOption{academicons}{\@academiconstrue}
%% v1.1.3: Choice of round/square photo
\newif\if@normalphoto
\DeclareOption{normalphoto}{\@normalphototrue}
\DeclareOption*{\PassOptionsToClass{\CurrentOption}{extarticle}}
\newif\if@raggedtwoe
\DeclareOption{ragged2e}{\@raggedtwoetrue}
\ProcessOptions\relax

\LoadClass{extarticle}

\RequirePackage[margin=2cm]{geometry}
\RequirePackage{fontawesome}
\RequirePackage{ifxetex,ifluatex}
\RequirePackage{scrlfile}

%% v1.1.5: added for convenience
\newif\ifxetexorluatex
\ifxetex
  \xetexorluatextrue
\else
  \ifluatex
    \xetexorluatextrue
  \else
    \xetexorluatexfalse
  \fi
\fi

%% v1.1: Optionally load academicons
%% v1.1.5: Handle different versions of academicons
\if@academicons
  \ifxetexorluatex
    \RequirePackage{fontspec}
    %% academicons in TL2018 doesn't require
    %% Academicons to be installed in OS fonts
    %% so can be loaded directly
    \@ifl@t@r\fmtversion{2018/04/01}{%
      \RequirePackage{academicons}
    }{%
      % TL2017
      \@ifl@t@r\fmtversion{2017/04/01}{%
        \@ifpackagelater{academicons}{2018/03/01}{%
          \RequirePackage{academicons}
        }{%
          \let\ori@newfontfamily\newfontfamily%
          \renewcommand{\newfontfamily}[2]{}
          \RequirePackage{academicons}
          \let\newfontfamily\ori@newfontfamily
          \newfontfamily{\AI}{academicons.ttf}
        }
      }{% TL2016 requires the package to be loaded before
        % the version can be checked. Only added because
        % Overleaf v1 still runs TL2016; will be removed
        % when v1 is completely retired.
          \let\ori@newfontfamily\newfontfamily%
          \renewcommand{\newfontfamily}[2]{}
          \RequirePackage{academicons}
          \let\newfontfamily\ori@newfontfamily
          \newfontfamily{\AI}{academicons.ttf}
      }
    }
  \else
    \ClassError{AltaCV}{academicons unsupported by latex or pdflatex. Please compile with xelatex or lualatex}{Please compile with xelatex or lualatex to use the academicons option}
  \fi
\fi

\if@raggedtwoe
  \RequirePackage[newcommands]{ragged2e}
\fi

\RequirePackage{xcolor}

\colorlet{accent}{blue!70!black}
\colorlet{heading}{black}
\colorlet{emphasis}{black}
\colorlet{body}{black!80!white}
\newcommand{\itemmarker}{{\small\textbullet}}
\newcommand{\ratingmarker}{\faCircle}

\RequirePackage{tikz}
\usetikzlibrary{arrows}
\RequirePackage[skins]{tcolorbox}
\RequirePackage{enumitem}
\setlist{leftmargin=*,labelsep=0.5em,nosep,itemsep=0.25\baselineskip,after=\vskip0.25\baselineskip}
\setlist[itemize]{label=\itemmarker}
\RequirePackage{graphicx}
\RequirePackage{etoolbox}
\RequirePackage{dashrule}
\RequirePackage{multirow,tabularx}
\RequirePackage{changepage}
% \RequirePackage{marginfix}

\setlength{\parindent}{0pt}
\newcommand{\divider}{\textcolor{body!30}{\hdashrule{\linewidth}{0.6pt}{0.5ex}}\medskip}

\newenvironment{fullwidth}{%
  \begin{adjustwidth}{}{\dimexpr-\marginparwidth-\marginparsep\relax}}
  {\end{adjustwidth}}

\newcommand{\emailsymbol}{\faAt}
\newcommand{\phonesymbol}{\faPhone}
\newcommand{\homepagesymbol}{\faChain}
\newcommand{\locationsymbol}{\faMapMarker}
\newcommand{\linkedinsymbol}{\faLinkedin}
\newcommand{\twittersymbol}{\faTwitter}
\newcommand{\githubsymbol}{\faGithub}
\newcommand{\orcidsymbol}{\aiOrcid}
\newcommand{\mailsymbol}{\faEnvelope}

\newcommand{\printinfo}[2]{\mbox{\textcolor{accent}{\normalfont #1}\hspace{0.5em}#2\hspace{2em}}}

\newcommand{\name}[1]{\def\@name{#1}}
\newcommand{\tagline}[1]{\def\@tagline{#1}}
\newcommand{\photo}[2]{\def\@photo{#2}\def\@photodiameter{#1}}
\newcommand{\email}[1]{\printinfo{\emailsymbol}{#1}}
\newcommand{\mailaddress}[1]{\printinfo{\mailsymbol}{#1}}
\newcommand{\phone}[1]{\printinfo{\phonesymbol}{#1}}
\newcommand{\homepage}[1]{\printinfo{\homepagesymbol}{#1}}
\newcommand{\twitter}[1]{\printinfo{\twittersymbol}{#1}}
\newcommand{\linkedin}[1]{\printinfo{\linkedinsymbol}{#1}}
\newcommand{\github}[1]{\printinfo{\githubsymbol}{#1}}
\newcommand{\orcid}[1]{\printinfo{\orcidsymbol}{#1}}
\newcommand{\location}[1]{\printinfo{\locationsymbol}{#1}}

\newcommand{\personalinfo}[1]{\def\@personalinfo{#1}}

\newcommand{\makecvheader}{%
  \begingroup
    \ifdef{\@photodiameter}{\begin{minipage}{\dimexpr\linewidth-\@photodiameter-2em}}{}%
    \raggedright\color{emphasis}%
    {\Huge\bfseries\MakeUppercase{\@name}\par}
    \medskip
    {\large\bfseries\color{accent}\@tagline\par}
    \medskip
    {\footnotesize\bfseries\@personalinfo\par}
    \ifdef{\@photodiameter}{%
    \end{minipage}\hfill%
    \begin{minipage}{\@photodiameter}
    \if@normalphoto
      \includegraphics[width=\linewidth]{\@photo}
    \else
      \tikz\path[fill overzoom image={\@photo}]circle[radius=0.5\linewidth];
    \fi%
    \end{minipage}\par}{}%
  \endgroup
  %\medskip
}

\renewenvironment{quote}{\color{accent}\itshape\large}{\par}

\newcommand{\cvsection}[2][]{%
  \bigskip%
  \ifstrequal{#1}{}{}{\marginpar{\vspace*{\dimexpr1pt-\baselineskip}\raggedright\input{#1}}}%
  {\color{heading}\LARGE\bfseries\MakeUppercase{#2}}\\[-1ex]%
  {\color{heading}\rule{\linewidth}{2pt}\par}\medskip
}

\newcommand{\cvsubsection}[1]{%
  \smallskip%
  {\color{emphasis}\large\bfseries{#1}\par}\medskip
}

% v1.1.4: fixes inconsistent font size
\newcommand{\cvevent}[4]{%
  {\large\color{emphasis}#1\par}
  \smallskip\normalsize
  \ifstrequal\textbf{#2}{}{}{
  \textbf{\color{accent}#2}
  \smallskip}%
  \ifstrequal{#3}{}{}{{\small\makebox[0.5 \linewidth][l]{\hfill\faCalendar\hspace{0.5em}#3}}}%
  \ifstrequal{#4}{}{}{{\small\makebox[0.5\linewidth][l]{\hfill\faGraduationCap\hspace{0.5em}#4}}}\par
  \medskip\normalsize
}

\newcommand{\cvevents}[4]{%
  {\large\color{emphasis}#1\par}
  \smallskip\normalsize
  \ifstrequal\textbf{#2}{}{}{
  \textbf{\color{accent}#2}\par
  \smallskip}
  \ifstrequal{#3}{}{}{{\small\makebox[0.5 \linewidth][l]{\faCalendar\hspace{0.5em}#3}}}%
  \ifstrequal{#4}{}{}{{\small\makebox[0.5\linewidth][l]{\hfill\faGraduationCap\hspace{0.5em}#4}}}\par
  \medskip\normalsize
}

\newcommand{\cvachievement}[3]{%
  \begin{tabularx}{\linewidth}{@{}p{2em} @{\hspace{1ex}} >{\raggedright\arraybackslash}X@{}}
  \multirow{2}{*}{\Large\color{accent}#1} & \bfseries\textcolor{emphasis}{#2}\\
  & #3
  \end{tabularx}%
  \smallskip
}

\newcommand{\cvtag}[1]{%
  \tikz[baseline]\node[anchor=base,draw=body!30,rounded corners,inner xsep=1ex,inner ysep =0.75ex,text height=1.5ex,text depth=.25ex]{#1};
}

\newcommand{\cvskill}[2]{%
\textcolor{emphasis}{\textbf{#1}}\hfill
\foreach \x in {1,...,5}{%
  \space{\ifnumgreater{\x}{#2}{\color{body!30}}{\color{accent}}\ratingmarker}}\par%
}

% Adapted from @Jake's answer at http://tex.stackexchange.com/a/82729/226
\newcommand{\wheelchart}[4][0]{%
    \begingroup\centering
    \def\innerradius{#3}%
    \def\outerradius{#2}%
    % Calculate total
    \pgfmathsetmacro{\totalnum}{0}%
    \foreach \value/\colour/\name in {#4} {%
        \pgfmathparse{\value+\totalnum}%
        \global\let\totalnum=\pgfmathresult%
    }%
    \begin{tikzpicture}

      % Calculate the thickness and the middle line of the wheel
      \pgfmathsetmacro{\wheelwidth}{\outerradius-\innerradius}
      \pgfmathsetmacro{\midradius}{(\outerradius+\innerradius)/2}
      \pgfmathsetmacro{\totalrot}{-90 + #1}

      % Rotate so we start from the top
      \begin{scope}[rotate=\totalrot]

      % Loop through each value set. \cumnum keeps track of where we are in the wheel
      \pgfmathsetmacro{\cumnum}{0}
      \foreach \value/\width/\colour/\name in {#4} {
            \pgfmathsetmacro{\newcumnum}{\cumnum + \value/\totalnum*360}

            % Calculate the percent value
            \pgfmathsetmacro{\percentage}{\value/\totalnum*100}
            % Calculate the mid angle of the colour segments to place the labels
            \pgfmathsetmacro{\midangle}{-(\cumnum+\newcumnum)/2}

            % This is necessary for the labels to align nicely
            \pgfmathparse{
               (-\midangle>180?"west":"east")
            } \edef\textanchor{\pgfmathresult}
            \pgfmathparse{
               (-\midangle>180?"flush left":"flush right")
            } \edef\textalign{\pgfmathresult}
            \pgfmathsetmacro\labelshiftdir{1-2*(-\midangle<180)}

            % Draw the color segments. Somehow, the \midrow units got lost, so we add 'pt' at the end. Not nice...
            \filldraw[draw=white,fill=\colour] (-\cumnum:\outerradius) arc (-\cumnum:-(\newcumnum):\outerradius) --
            (-\newcumnum:\innerradius) arc (-\newcumnum:-(\cumnum):\innerradius) -- cycle;

            % Draw the data labels
            \draw  [*-,thin,emphasis] node [append after command={(\midangle:\midradius pt) -- (\midangle:\outerradius + 1ex) -- (\tikzlastnode)}] at (\midangle:\outerradius + 1ex) [xshift=\labelshiftdir*0.5cm,inner sep=1ex, outer sep=0pt, text width=\width,anchor=\textanchor,align=\textalign,font=\small,text=body]{\name};
            % Set the old cumulated angle to the new value
            \global\let\cumnum=\newcumnum
        }
      \end{scope}
%      \draw[gray] (0,0) circle (\outerradius) circle (\innerradius);
    \end{tikzpicture}\par
    \endgroup
}

\newcommand{\cvref}[3]{%
  \smallskip
  \textcolor{emphasis}{\textbf{#1}}\par
  \begin{description}[font=\color{accent},style=multiline,leftmargin=1.35em]
  \item[\normalfont\emailsymbol] #2
  \item[\small\normalfont\mailsymbol] #3
  \end{description}
%   \medskip
}

\newenvironment{cvcolumn}[1]{\begin{minipage}[t]{#1}\raggedright}{\end{minipage}}

\RequirePackage[backend=biber,style=authoryear,sorting=ydnt]{biblatex}
%% For removing numbering entirely when using a numeric style
% \setlength{\bibhang}{1em}
% \DeclareFieldFormat{labelnumberwidth}{\makebox[\bibhang][l]{\itemmarker}}
% \setlength{\biblabelsep}{0pt}
\defbibheading{pubtype}{\cvsubsection{#1}}
\renewcommand{\bibsetup}{\vspace*{-\baselineskip}}
\AtEveryBibitem{\makebox[\bibhang][l]{\itemmarker}}
\setlength{\bibitemsep}{0.25\baselineskip}

% v1.1.2: make it easier to add a sidebar aligned with top of next page
\RequirePackage{afterpage}
\newcommand{\addsidebar}[2][]{\marginpar{%
  \ifstrequal{#1}{}{}{\vspace*{#1}}%
  \input{#2}}%
}
\newcommand{\addnextpagesidebar}[2][]{\afterpage{\addsidebar[#1]{#2}}}

\newcommand{\cvproject}[1]{%
  {\textbf{\color{accent}#1}\par}
  \smallskip\normalsize
}

\AtBeginDocument{%
  \pagestyle{empty}
  \color{body}
  \raggedright
}
